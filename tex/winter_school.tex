\documentclass[aspectratio=169,nototalframenumber]{beamer}
\usetheme[nosectiontitlepage]{uibk}

\usepackage{tikz}
\usetikzlibrary{shapes}

\usepackage{graphicx}
\graphicspath{{fig/}}

\input{macros/mit-new.tex}
\usepackage{tikz}
\usetikzlibrary{arrows, patterns}

\RequirePackage{xcolor}
\definecolor{color1}{HTML}{7D0025}
\definecolor{color2}{HTML}{AB3816}
\definecolor{color3}{HTML}{C66D2E}
\definecolor{color4}{HTML}{DA9C5B}
\definecolor{color5}{HTML}{EAC486}
\definecolor{color6}{HTML}{F9E7AE}
\definecolor{color7}{HTML}{FFFFC8}

\newcommand{\legend}[2]{
  \begin{scope}[xshift=#1, yshift=#2]
    \draw[draw] (0,0) rectangle ++(3,-1);
    \draw[draw, fill=color1] (0.1,-0.1) rectangle ++(0.4,-0.2);
    \draw[anchor=west] (0.5, -0.2) node {\scriptsize Tensorflow time};
    \draw[draw, fill=color4] (0.1,-0.4) rectangle ++(0.4,-0.2);
    \draw[anchor=west] (0.5, -0.5) node {\scriptsize CPU time};
    \draw[draw, preaction={fill,color7}, pattern=north east lines] (0.1,-0.7) rectangle ++(0.4,-0.2);
    \draw[anchor=west] (0.5, -0.8) node {\scriptsize Enclave penalty};
  \end{scope}
}


\title{Enclave-NN}
\author{Alexander Schl\"ogl}

\begin{document}

\uibktitlepage{}

\begin{frame}[label=basics]
    \frametitle{Introduction}

    Neural Nets are universal approximators\cite{hornik1991approximation}.

	\vspace{4ex}

	\begin{block}{Technical Details}
    	\begin{itemize}
        	\item Forward pass is a series of algebraic operations
        	\item Fully defined by their architecture and weights (parameters)
    	\end{itemize}
	\end{block}

	We focus solely on inference phase, so NNs for us are static functions.
\end{frame}

\begin{frame}
    \frametitle{Problem Statement}

	Monetization requires keeping parameters private.
	\vspace{3ex}
	\pause

	\begin{block}{Online Oracles}
    	\begin{itemize}
        	\item allow continuous monetization
        	\item require sharing of data for inference
        	\item require provider's infrastructure
    	\end{itemize}
	\end{block}

	\pause

	\begin{alertblock}{Idea}
    	Use Trusted Execution Environments (TEEs) to hide parameters during inference
	\end{alertblock}
\end{frame}

\begin{frame}
    \frametitle{Approach}

	\begin{figure}
    	\begin{tikzpicture}
        	\mitnetsummary{2}

			\onslide<2->{
    			\draw[draw=uibkgray,fill=uibkgray] (6,-.0) rectangle ++(.5,-.8);
    			\draw[draw=uibkgray,fill=uibkgray] (5.75,-.8) -- ++(1,0) -- ++(-.5,-.4) -- cycle;

            	\mitnetsummary{-2}
			}

			\onslide<2>{
				\draw[text=uibkblue!100,draw=uibkblue,fill=uibkblue!30,fill opacity=0.7,line width=0.5mm] (9.96,-1.6) coordinate (teebox) rectangle (12.5, -4);
                \draw node[text=uibkblue, anchor=north west] at (teebox) {\small TEE};
            }
			\onslide<3>{
				\draw[text=uibkblue!100,draw=uibkblue,fill=uibkblue!30,fill opacity=0.7,line width=0.5mm] (7.465,-1.6) coordinate (teebox) rectangle (12.5, -4);
                \draw node[text=uibkblue, anchor=north west] at (teebox) {\small TEE};
            }
			\onslide<4>{
				\draw[text=uibkblue!100,draw=uibkblue,fill=uibkblue!30,fill opacity=0.7,line width=0.5mm] (-.06,-1.6) coordinate (teebox) rectangle (12.5, -4);
                \draw node[text=uibkblue, anchor=north west] at (teebox) {\small TEE};
            }
		\end{tikzpicture}
	\end{figure}
\end{frame}

\begin{frame}
    \frametitle{Evaluation Method}

	\begin{enumerate}
		\item Split NN
		\item Compile TEE and native code
		\item Measure inference time on single input
		\item Separate CPU impact from TEE impact
	\end{enumerate}

	\vspace{1cm}
	Repeat for every layer in NN
\end{frame}

\begin{frame}
    \frametitle{Results}

\begin{tikzpicture}[>=stealth,yscale=0.8]
  \draw (0,0) -- (\mitnetwidth,0);
  \foreach \x in \mitxticks {
    \draw (\x,0) -- (\x, -0.1);
  }

  \draw[->] (0,0) -- (0,\ymax+0.1);
  \foreach \y in \yticks {
    \draw (-0.1,\y) -- (0.1,\y);
  }
  \ylabels{-.5}
  
  \mitnetsummary{-0.2}

  \mitnewsplita
  \mitnewsplitb
  \mitnewsplitc
  \mitnewsplitd
  \mitnewsplite
  \mitnewsplitf
  \mitnewsplitg
  \mitnewsplith
  \mitnewspliti
  \mitnewsplitj
  \mitnewsplitba
  \mitnewsplitbb
  \mitnewsplitbc
  \mitnewsplitbd
  \mitnewsplitbe
  \mitnewsplitbf
  \mitnewsplitbg
  \mitnewsplitbh
  \mitnewsplitbi
  \mitnewsplitbj
  \mitnewsplitca
  \mitnewsplitcb
  \mitnewsplitcc
  \mitnewsplitcd
  \mitnewsplitce

  \legend{9cm}{6.3cm}

\end{tikzpicture}
\end{frame}

\begin{frame}
    \frametitle{Open Questions}
\end{frame}

\end{document}
